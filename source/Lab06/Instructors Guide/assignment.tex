\documentclass[12pt,a4paper]{article}
% AUTHOR: Rafael Belchior
% Thanks to Prof. RUI SANTOS CRUZ for providing the template
%
\usepackage{helvet}
\renewcommand{\familydefault}{\sfdefault}
\usepackage{a4wide}
\usepackage{ucs}
\usepackage[utf8x]{inputenc}
\usepackage{amsmath}

%%%%%%%%%%%%%%%%%%%%%%%%%%%%%%%%%%%%%%%%%%%%%%%%%%%%%%%%%%%%%%%%%%%%%%%%%%%%%%%%%%%
% SELECT ONE OF THE FOLLOWING PACKAGES FOR THE LANGUAGE 
\usepackage[portuges]{babel}
% \usepackage[portuges]{babel}
%%%%%%%%%%%%%%%%%%%%%%%%%%%%%%%%%%%%%%%%%%%%%%%%%%%%%%%%%%%%%%%%%%%%%%%%%%%%%%%%%%
\usepackage{subfig}
\usepackage{graphicx}
\usepackage{hyperref}
\usepackage{cite}
\usepackage[absolute]{textpos}
\usepackage{tabularx}
\usepackage{tabulary}
\usepackage{fancyhdr}
\usepackage[table]{xcolor}
\pagestyle{fancy}
\headsep=50pt
\setlength{\headheight}{50pt}
\usepackage{listings}
\usepackage{minted}
\definecolor{LightGray}{rgb}{0.95, 0.95, 0.95}
\definecolor{darkblue}{rgb}{0.0,0.0,0.6}
\definecolor{editorOcher}{rgb}{1, 0.5, 0}

% Clever Referencing of document parts
\usepackage{cleveref}

\lstdefinestyle{commandline} {%
language={[WinXP]command.com},
breaklines=true,
%aboveskip=\baselineskip,
belowskip=\baselineskip,
showstringspaces=false,
backgroundcolor=\color{LightGray},
basicstyle=\small\color{black}\ttfamily,
showstringspaces=false,
keywordstyle=\color{cyan}\bfseries,
stringstyle=\color{cyan}\ttfamily,
commentstyle=\color{green}\itshape,
moredelim=[s][\color{blue}\bfseries]{C:}{\>}
}

\lstdefinestyle{Bash} {%
language=bash,
breaklines=true,
belowskip=\baselineskip,
backgroundcolor=\color{LightGray},
showstringspaces=false,
keywordstyle=\color{black}\bfseries,
basicstyle=\small\color{black}\ttfamily,
stringstyle=\color{editorOcher}\ttfamily,
commentstyle=\color{cyan}\itshape,
otherkeywords={xcode-select, mkdir,rm},
moredelim=[s][\color{red}]{~$},
literate={~} {$\sim$}{1}
}
%%%%%%%%%%%%%%%%%%%%%%%%%%%%%%%%%%%%%%%%%%%%%%%%%%%%%%%%%%%%%%%%%%%%%%%%%%%%%%%%%%%
% PLEASE FILL THE ADEQUATE DATA IN THE TABLE REPLACING
% THE VALUES EXEMPLIFIED
\lhead{}
{\renewcommand{\arraystretch}{1.1}
\fancyhead[C]{\begin{tabularx}{1.0\textwidth}{|l|X|l|l|}
\hline
% In the following line change Course Name: PPIII, PPB
\textbf{EB 20/21} & \textbf{Enterprise Blockchain Technologies} & \textbf{Number:}  &  6\\
\hline
% In the following line insert your Name and IST ID
\multicolumn{2}{|l|}{Module II - Hyperledger Fabric} & \textbf{Issue Date:}  & - \\
\hline
% In the following line insert the Activity CODE and Title (abridged)
%\textbf{WP n.} (99) & (Subject) & \textbf{Group:} & (99) \\
\multicolumn{2}{|l|}{Hyperledger Fabric: Full-Stack dApp} & \textbf{Due Date:} &  - \\
\hline
\end{tabularx}}
\rhead{}

%%%%%%%%%%%%%%%%%%%%%%%%%%%%%%%%%%%%%%%%%%%%%%%%%%%%%%%%%%%%%%%%%%%%%%%%%%%%%%%%%%%
% DO NOT CHANGE THIS BLOCK
\begin{document}
\textblockorigin{-34pt}{-12pt}
\begin{textblock*}{10cm}(2cm,1cm)
\includegraphics[width=6cm]{hyperledger.png}
\end{textblock*}
%%%%%%%%%%%%%%%%%%%%%%%%%%%%%%%%%%%%%%%%%%%%%%%%%%%%%%%%%%%%%%%%%%%%%%%%%%%%%%%%%%%,sdist2017

\section*{Instructors Guide}
The majority of this laboratory was inspired by the implementation of JusticeChain\cite{belchior2019_audits}. Part of this laboratory was inspired in sources elaborated by KC Tam\footnote{https://medium.com/@kctheservant}.

\section{Exercises}


\subsubsection*{The \texttt{network.sh} script can be reproduced step by step\footnote{https://medium.com/@kctheservant/add-a-peer-to-an-organization-in-test-network-hyperledger-fabric-v2-2-4a08cb901c98}\footnote{https://hyperledger-fabric.readthedocs.io/en/release-2.2/tutorial/commercial_paper.html}. Reproduce the process described on the B4S network.}

\subsubsection*{Suppose that we want to add a new peer node to the network. Referring to Figure 3, which steps are to be taken?}

A:

Step 2: generate crypto material for the peer
Step 4: create a docker-compose file containing this peer
Step 5c: join this new peer to the existing channel
Step 6b: install chaincode to that peer

\subsubsection*{In the deployCC.sh script, what is the difference between chaincodeInvoke and chaincodeInvokeInit?}

A: chaincodeInvokeInit is a widely used convention on Fabric 1.4 (as every instance of the blockchain needed a transaction to initialize the ledger). In Fabric 2.x, there is no need for this step. In practice, chaincodeInvoke and chaincodeInvokeInit are used to test if there is no error with the deployment. They call different methods on the org.b4s.quc smart contract. However, the peer command called is precisely the same (peer chaincode invoke).

\subsubsection*{What are some dangers of using the web app as a centralized party to access the blockchain?}
A: The transactions are issued accordingly to the arguments passed by the web app. If the web app is not open-source, we may not know what is being given to the blockchain.


\subsubsection*{What are the differences of this project to a production environment?}
A: Some of them: A back office may exist, where organizations could add/remove channels, participants, and peer nodes, in a distributed setting. The frontend could dynamically load user credentials: users could use a card with their credentials to sign credentials without the system having to hold them. The crypto material is generated by different organizations and not shared. There might be various web applications and interfaces, depending on the organizations. There is chaincode access control (for example, only specific users may access certain functions).
%%%%%%%%%%%%%%%%%%%%%%%%%%%%%%%%%%%%%%%%%%%%%%%%%%%%%%%%%%%%%%%%
\bibliographystyle{IEEEtran}
\bibliography{lab.bib}

\end{document}                             % The required last line
