\documentclass[12pt,a4paper]{article}
% AUTHOR: Rafael Belchior
% Thanks to Prof. RUI SANTOS CRUZ for providing the template
%
\usepackage{helvet} 
\renewcommand{\familydefault}{\sfdefault}
\usepackage{a4wide}
\usepackage{ucs}
\usepackage[utf8x]{inputenc}
\usepackage{amsmath}

%%%%%%%%%%%%%%%%%%%%%%%%%%%%%%%%%%%%%%%%%%%%%%%%%%%%%%%%%%%%%%%%%%%%%%%%%%%%%%%%%%%
% SELECT ONE OF THE FOLLOWING PACKAGES FOR THE LANGUAGE 
\usepackage[portuges]{babel}
% \usepackage[portuges]{babel}
%%%%%%%%%%%%%%%%%%%%%%%%%%%%%%%%%%%%%%%%%%%%%%%%%%%%%%%%%%%%%%%%%%%%%%%%%%%%%%%%%%
\usepackage{subfig}
\usepackage{graphicx}
\usepackage{hyperref}
\usepackage{cite}
\usepackage[absolute]{textpos}
\usepackage{tabularx} 
\usepackage{tabulary}                 
\usepackage{fancyhdr}
\usepackage[table]{xcolor}
\pagestyle{fancy}
\headsep=50pt
\setlength{\headheight}{50pt}
\usepackage{listings}
\usepackage{minted}
\definecolor{LightGray}{rgb}{0.95, 0.95, 0.95}
\definecolor{darkblue}{rgb}{0.0,0.0,0.6}
\definecolor{editorOcher}{rgb}{1, 0.5, 0}

% Clever Referencing of document parts
\usepackage{cleveref}

\lstdefinestyle{commandline} {%
language={[WinXP]command.com},
breaklines=true,
%aboveskip=\baselineskip,
belowskip=\baselineskip,
showstringspaces=false,
backgroundcolor=\color{LightGray},
basicstyle=\small\color{black}\ttfamily,
showstringspaces=false,
keywordstyle=\color{cyan}\bfseries,
stringstyle=\color{cyan}\ttfamily,
commentstyle=\color{green}\itshape,
moredelim=[s][\color{blue}\bfseries]{C:}{\>}
}

\lstdefinestyle{Bash} {%
language=bash,
breaklines=true,
belowskip=\baselineskip,
backgroundcolor=\color{LightGray},
showstringspaces=false,
keywordstyle=\color{black}\bfseries,
basicstyle=\small\color{black}\ttfamily,
stringstyle=\color{editorOcher}\ttfamily,
commentstyle=\color{cyan}\itshape,
otherkeywords={xcode-select, mkdir,rm},
moredelim=[s][\color{red}]{~$},
literate={~} {$\sim$}{1}
}
%%%%%%%%%%%%%%%%%%%%%%%%%%%%%%%%%%%%%%%%%%%%%%%%%%%%%%%%%%%%%%%%%%%%%%%%%%%%%%%%%%%
% PLEASE FILL THE ADEQUATE DATA IN THE TABLE REPLACING
% THE VALUES EXEMPLIFIED
\lhead{}
{\renewcommand{\arraystretch}{1.1}
\fancyhead[C]{\begin{tabularx}{1.0\textwidth}{|l|X|l|l|}
\hline 
% In the following line change Course Name: PPIII, PPB
\textbf{EB 20/21} & \textbf{Enterprise Blockchain Technologies} & \textbf{Number:}  &  7\\
\hline
% In the following line insert your Name and IST ID
\multicolumn{2}{|l|}{Module II - Hyperledger Fabric} & \textbf{Issue Date:}  & - \\ 
\hline
% In the following line insert the Activity CODE and Title (abridged)
%\textbf{WP n.} (99) & (Subject) & \textbf{Group:} & (99) \\
\multicolumn{2}{|l|}{Advanced Topics: Hyperledger Umbra} & \textbf{Due Date:} &  - \\ 
\hline
\end{tabularx}}
\rhead{}

%%%%%%%%%%%%%%%%%%%%%%%%%%%%%%%%%%%%%%%%%%%%%%%%%%%%%%%%%%%%%%%%%%%%%%%%%%%%%%%%%%%
% DO NOT CHANGE THIS BLOCK
\begin{document}
\textblockorigin{-34pt}{-12pt}
\begin{textblock*}{10cm}(2cm,1cm)
\includegraphics[width=6cm]{hyperledger.png}
\end{textblock*}
%%%%%%%%%%%%%%%%%%%%%%%%%%%%%%%%%%%%%%%%%%%%%%%%%%%%%%%%%%%%%%%%%%%%%%%%%%%%%%%%%%%,sdist2017

\section*{Instructors Guide}
The Hyperledger Umbra (Umbra)\footnote{https://github.com/hyperledger-labs/umbra} project, is a project initiated as an Hyperledger Internship. This laboratory has been created with the cooperation of Raphael Rosa, creator, and maintainer of Umbra. Please, mind the official documentation page for eventual questions\footnote{https://umbra-labs.readthedocs.io/en/latest/intro.html}, and the official channels to communicate issues\footnote{https://github.com/hyperledger-labs/umbra/issues}. 


\section{Exercises}


\subsubsection*{What are the advantages and disadvantages of Umbra, regarding the evolution of enterprise blockchain technologies?}

A: Allows to empirically test blockchain properties such as scalability, consensus resiliency, and privacy. This allows fixing issues that are difficult to verify in an ad-hoc way. As the disadvantages, to integrate a new blockchain, it requires more effort than an ad-hoc evaluation.

\subsubsection*{Creating a topology with Umbra}

Please refer to the official repository.



%%%%%%%%%%%%%%%%%%%%%%%%%%%%%%%%%%%%%%%%%%%%%%%%%%%%%%%%%%%%%%%%
\bibliographystyle{IEEEtran}
\bibliography{lab.bib}

\end{document}                             % The required last line
